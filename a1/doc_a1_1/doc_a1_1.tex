\documentclass[12pt, twoside]{article}
\usepackage{amsmath}
\usepackage{amssymb}
\usepackage[colorlinks=true, linkcolor=black]{hyperref} % Links
\usepackage{makeidx} % Indexierung
\usepackage{siunitx}
\usepackage[ngerman]{babel} % deutsche Sonderzeichen
\usepackage[utf8]{inputenc}
\usepackage{geometry} % Dokumentendesign wie Seiten- oder Zeilenabstand bestimmen
\usepackage[toc,page]{appendix}

% Graphiken
\usepackage{tikz}
\usepackage{pgfplots}
\usepackage{pgfcore}
\usepackage{pgfopts}
\usepackage{pgfornament}
\usepackage{pgf}
\usepackage{ifthen}
\usepackage{booktabs}

% Tabellen
\usepackage{tabu}
\usepackage{longtable}
\usepackage{colortbl} % Tabellen faerben
\usepackage{multirow}
\usepackage{diagbox} % Tabellenzelle diagonal splitten

\usepackage{xcolor} % Farben
\usepackage[framemethod=tikz]{mdframed} % Hintergrunderstellung
\usepackage{enumitem} % Enumerate mit Buchstaben nummerierbar machen
\usepackage{pdfpages}
\usepackage{listings} % Source-Code darstellen
\usepackage{eurosym} % Eurosymbol
\usepackage[square,numbers]{natbib}
\usepackage{here} % figure an richtiger Stelle positionieren
\usepackage{verbatim} % Blockkommentare mit \begin{comment}...\end{comment}
\usepackage{ulem} % \sout{} (durchgestrichener Text)

% BibLaTex
\bibliographystyle{acm}

% Aendern des Anhangnamens (Seite und Inhaltsverzeichnis)
\renewcommand\appendixtocname{Anhang}
\renewcommand\appendixpagename{Anhang}

% mdframed Style
\mdfdefinestyle{codebox}{
	linewidth=2.5pt,
	linecolor=codebordercolor,
	backgroundcolor=codecolor,
	shadow=true,
	shadowcolor=black!40!white,
	fontcolor=black,
	everyline=true,
}

% Seitenabstaende
\geometry{left=15mm,right=15mm,top=15mm,bottom=20mm}

% TikZ Bibliotheken
\usetikzlibrary{
    arrows,
    arrows.meta,
    decorations,
    backgrounds,
    positioning,
    fit,
    petri,
    shadows,
    datavisualization.formats.functions,
    calc,
    shapes,
    shapes.multipart
}

\pgfplotsset{width=7cm,compat=1.15}

\definecolor{codecolor}{HTML}{EEEEEE}
\definecolor{codebordercolor}{HTML}{CCCCCC}

% Standardeinstellungen fuer Source-Code
\lstset{
    language=C,
    breaklines=true,
    keepspaces=true,
    keywordstyle=\bfseries\color{green!70!black},
    basicstyle=\ttfamily\color{black},
    commentstyle=\itshape\color{purple},
    identifierstyle=\color{blue},
    stringstyle=\color{orange},
    showstringspaces=false,
    rulecolor=\color{black},
    tabsize=2,
    escapeinside={\%*}{*\%},
}

%\input{libuml}
%\input{liberm}

\begin{document}

\begin{center}
  \Huge{Praktikum Künstliche Intelligenz} \\
  \huge{Team D Rot} \\
  \Large{Aufgabe 1.1} \\
\end{center}

\begin{enumerate}[label={\textbf{\alph*)}}]

  \item \textbf{Entscheiden Sie sich f\"ur eine sinnvolle
          Fitnessfunktion und eine sinnvolle
          Abbruchbedingung f\"ur den Algorithmus}

        Sei $\mathbb{K} \subseteq \mathbb{N}:=\{1,2,3,4\}$,
        $\mathbb{V} \subseteq \mathbb{Z} := \{-50,10,100,
        200\}$ und die Relation aller Gene $G \subseteq
        \mathbb{K} \times \mathbb{V} := \{(1,100), (2,10),
        (3, -50),(4, 200)\}$. Gesucht ist eine Folge von
        acht Genen $S_{max} \in \mathbb{S}^8$
        ($\mathbb{S}^8$ die Menge aller Gensequenzen mit
        l\"ange acht und jedem Element in $G$), die die
        h\"ochste Zahl bilden:
        $S_{max} = max(\Sigma_{j=1}^{8} G[S_{ij}]), S_{i}
        \in \mathbb{S}^8, G[x] \in \mathbb{V}, x \in
        \mathbb{K}$,
        wobei in diesem Fall
        $\Sigma_{j=1}^{8} G[S_{ij}],S_{i} \in \mathbb{S}^8$
        unsere Fitnessfunktion darstellt.

        Eine sinnvole Abbruchbedingung w\"are, wenn
        $S_{max}$ gefunden ist.

  \item \textbf{Wie lautet der genetische Code des
          perfekten Individuums?}

        $S_{max} = [4, 4, 4, 4, 4, 4, 4, 4]$

  \item \textbf{Wie hoch ist die Wahrscheinlichkeit, dass
          dieses Individuum in unser zuf\"allig generierten
          ersten Generation bereits ein- oder mehrmals
          enthalten ist?}

        Die Anzahl an Individuen unserer ersten Population
        $\mathbb{P}$ betr\"agt 1000.

        Die Wahrscheinlichkeit $S_{max} \in \mathbb{P}$
        liegt damit bei:

        \begin{align*}
          p(S_{max}) &= \frac{1}{|G|^{8}} = \frac{1}{4^8}
                     = \frac{1}{65536} \\
          p(S_{max} \in \mathbb{P}) &= \frac{|P|}{|G|^{8}}
                     =\frac{1000}{4^8} = \frac{1000}{65536}
        \end{align*}

  \item[\textbf{e)}] \textbf{Welche Gefahr best\"unde, wenn
          wir auf den Mutationsschritt verzichten
          w\"urden?}

        Bei einer schlechten ersten Generation besteht die
        Gefahr, dass, dadurch, dass keine Mutation
        angewendet wird, die folgenden Generationen nicht
        f\"ahig sind, $S_{max}$ zu bilden. Gelte
        beispielsweise: $\forall S_i \in \mathbb{P}, S_{i1}
        \neq 4$, k\"onnte $S_{max}$ nicht gebildet werden,
        da nie eine 4 an erster Stelle der Sequenz stehen
        kann.

  \item[\textbf{f)}] \textbf{Welche \"Anderung w\"urden Sie
          vornehmen, um den oben beschriebenen Algorithmus
          zu optimieren?}

        Wir w\"urden erst das Element aus $G$ suchen, dass
        maximal ist und dieses acht mal mit sich selbst
        addieren.

        Falls es ein genetischer Ansatz sein soll, w\"urden
        wir anstatt Single-Point-Crossover einen Ansatz
        verfolgen, bei dem wir zwei Individuuen sich so
        fortpflanzen lassen, dass ein neues Individuum
        entsteht, welches an jeder Stelle seiner Gensequenz
        das Maximum seiner Eltern hat. Jede Generation, die
        sich durch diesen Schritt halbieren w\"urde, wird
        mit neuen, zuf\"alligen Individuen
        angereichert.

\end{enumerate}

\end{document}
