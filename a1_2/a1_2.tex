\documentclass{beamer}

\usepackage{amsmath}
\usepackage{amssymb}

\begin{document}

\begin{frame}
  \begin{center}
    \Huge{Praktikum K\"unstliche Intelligenz} \\
    \vspace{1.5cm}
    \huge{Team D Rot} \\
    \vspace{1.5cm}
    \Large{Aufgabe 1.2}
  \end{center}
\end{frame}

\begin{frame}
  \frametitle{\textbf{a)} Wahl der Varianten/Optionen}
  Content.
\end{frame}

\begin{frame}
  \frametitle{\textbf{b)} Fitnessfunktion}

  Sei $S_{max}$ perfektes Individuum (Gensequenz). \\

  Distanzfunktion zwischen zwei Genen $x$ und $y$:
  \begin{align*}
    d(x,y) =
      \begin{cases}
        1 & \quad \text{falls } x = y \\
        0 & \quad \text{falls } x \neq y
      \end{cases}
  \end{align*}

  Fittnesfunktion eines Individuums $S$:

  \begin{align*}
    f(S) = \frac{\Sigma_{j=1}^{|S|} d(S_{j},S_{max_j})}
                {|S|}
  \end{align*}
  \textbf{Beispiel} \\
    \vspace{0.5cm}
    Sei $S_{max} := [ A, A, B ]$, $S := [A, C, A]$. \\
    \vspace{0.5cm}
    $f(S) = \frac{d(A,A) + d(C, A) + d(A, B)}{3}
           = \frac{1 + 0 + 0}{3} = \frac{1}{3}$
\end{frame}

\begin{frame}
  \frametitle{\textbf{c)} Gene}
  Content.
\end{frame}

\begin{frame}
  \frametitle{\textbf{d)} Einsch\"atzung zur Optimierung}
  Content.
\end{frame}

\begin{frame}
  \frametitle{\textbf{e)} In Realit\"at anwendbar?}
  Content.
\end{frame}

\end{document}
