\documentclass{beamer}

\usepackage{amsmath}
\usepackage{amssymb}

\begin{document}

\begin{frame}
  \begin{center}
    \Huge{Praktikum K\"unstliche Intelligenz} \\
    \vspace{1.5cm}
    \huge{Team D Rot} \\
    \vspace{1.5cm}
    \Large{Aufgabe 1.2}
  \end{center}
\end{frame}

\begin{frame}
  \frametitle{\textbf{a)} Wahl der Varianten/Optionen}
  \begin{itemize}
    \item \textbf{Selektor:} Random Rejection
    \item \textbf{Erklärung:} Wir wählen einen zufälliges Individuum unserer Population. Der Fitnesswert 		muss größer oder gleich  einem zweiten zufälligen Fitnesswert, welcher in einem Bereich zwischen dem 		schlechtesten und besten Fitnesswert der Population liegt, sein. 
   \item \textbf{Entscheidungsgrund:} Durch die zwei Zufallskomponenten, entfällt bei einer Selektion 			nach gewichteten Wahrscheinlichkeiten die Erstellung eines großen Pools.
   
  \end{itemize}
\end{frame}

\begin{frame}
  \frametitle{\textbf{a)} Wahl der Varianten/Optionen}
  \begin{itemize}
    \item \textbf{Mutator:} Swap-Mutator mit Random Resetting
    \item Wir wählen einen zufälliges Individuum unserer Population. Der Fitnesswert muss größer oder gleich  einem zweiten zufälligen Fitnesswert, welcher in einem Bereich zwischen dem schlechtesten und besten Fitnesswert der Population liegt, sein. 
   
   
  \end{itemize}
\end{frame}

\begin{frame}
  \frametitle{\textbf{b)} Fitnessfunktion}

  Sei $S_{max}$ perfektes Individuum (Gensequenz). \\

  Distanzfunktion zwischen zwei Genen $x$ und $y$:
  \begin{align*}
    d(x,y) =
      \begin{cases}
        1 & \quad \text{falls } x = y \\
        0 & \quad \text{falls } x \neq y
      \end{cases}
  \end{align*}

  Fittnesfunktion eines Individuums $S$:

  \begin{align*}
    f(S) = \frac{\Sigma_{j=1}^{|S|} d(S_{j},S_{max_j})}
                {|S|}
  \end{align*}
  \textbf{Beispiel} \\
    \vspace{0.5cm}
    Sei $S_{max} := [ A, A, B ]$, $S := [A, C, A]$. \\
    \vspace{0.5cm}
    $f(S) = \frac{d(A,A) + d(C, A) + d(A, B)}{3}
           = \frac{1 + 0 + 0}{3} = \frac{1}{3}$
\end{frame}

\begin{frame}
  \frametitle{\textbf{c)} Gene}
  Content.
\end{frame}

\begin{frame}
  \frametitle{\textbf{d)} Einsch\"atzung zur Optimierung}
  Content.
\end{frame}

\begin{frame}
  \frametitle{\textbf{e)} In Realit\"at anwendbar?}
  Content.
\end{frame}

\end{document}
 \item \textbf{Rekombinator:} Single-Point-Crossover
    \item \textbf{Mutator:}
      \begin{enumerate}
        \item Tauscht zuf\"allig zwei Gene des Individuums
        \item Iteriere Gene des Individuums. F\"ur jedes
              Gen 1\%-ige Chance, dass es mutiert (random)
      \end{enumerate}